%-*- coding: UTF-8 -*-
% Letter.tex
% Letetr and Symbol

% ---------------------- Introduction ----------------------
\documentclass{article}

\usepackage[OT2,OT1,T1]{fontenc}        % text encoding selection, may have problem in lualatex
\usepackage[greek,english]{babel}       % language, may have problem in lualatex. TODO: See the babel documentation. 
\usepackage[utf8]{inputenc}             % FIXME: input encoding?
\usepackage{textcomp}                   % provide special symbol command
\usepackage{fonttable}
\usepackage[normalem]{ulem}
\usepackage{setspace}
\newcommand\Emph{\textbf}
%\usepackage{txfonts}                   % may conflict with fontspec
%\usepackage{fontspec}

%\usepackage{ccfonts}
%\usepackage[no-math]{fontspec}

%\setmainfont{Garamond}                 % depend on fontspec, and do not contain greek letter and cyrillic. 
%\setsansfont{Arial}                    % depend on fontspec
%\setmonofont{Edwardian Script ITC}     % depend on fontspec

%\newfontfamily\inkfamily{Ink Free}     % set new command inkfamily to use Ink Free, only support by fontspec package. 
%\renewcommand\rmdefault{cmfib}         % change the rm family into cmfib

\setlength\lineskiplimit{2.5bp}         % lineskiplimit
\setlength\lineskip{2.5bp}              % lineskip

\newlength\mylen                        % define mylen
\setlength\mylen{2em}

\newsavebox\mybox                       % define mybox
\sbox\mybox{text text}

\title{Letter and Symbol}
\author{Wenren Muyan}
\date{\today}

% ---------------------- Introduction ----------------------

\begin{document}
    \maketitle

    \begin{abstract}
        Input letter and symbol. 
    \end{abstract}

    \tableofcontents

    \section{Letter}
        \subsection{Accents}
            \`o \phantom{50} \'o \phantom{50} \^o \phantom{50} \"o \newline
            \~o \phantom{50} \=o \phantom{50} \.o\phantom{50} \u{o} \newline
            \v{o} \phantom{50} \H{o} \phantom{50} \t{oo} \phantom{50} \r{o}\newline
            \c{o} \phantom{50} \d{o} \phantom{50} \b{o}\newline
            \AA \phantom{50} \aa \phantom{50} \AE \phantom{50} \ae\newline
            \OE \phantom{50} \oe \phantom{50} \SS \phantom{50} \ss\newline
            \IJ \phantom{50} \ij \phantom{50} \L \phantom{50} \l\newline
            \O \phantom{50} \o \phantom{50} \i \phantom{50} \j

        \subsection{Other Letters}
            \textgreek{abcde}\newline                   % would produce error in lualatex
            %\textrussian{abcde}
            % More details in https://ctan.math.utah.edu/ctan/tex-archive/macros/latex/contrib/babel-contrib/greek/usage.pdf
            {\fontencoding{OT2}\selectfont ABCDE}       % would produce error in lualatex
            % Input greek letter and cyrillic by using babel or fontenc macros. 
            % More details in http://latex3.github.io/help/documentation/cyrguide.pdf
            % NOTICE: Some letters need support all of the editor, input font and output font. 
            % TODO: More details of using babel are in http://www.leliseron.org/annexes/LaTeX/babel.pdf
            % FIXME: How to input Russian with babel? 
            % REVIEW: 
            % DEBUG:
            % NOTE:

        \subsection{Ligature}
            differ find flight difficult ruffle\newline
            dif{}fer f{}ind f\/light dif\/f\/icult ruffle

    \section{Punctuation}
        \subsection{Common Punctuation}
            , \quad . \quad ; \quad :\newline
            ! \quad ? \quad ` \quad '\newline
            ( \quad ) \quad [ \quad ]\newline
            \quad - \quad / \quad * \quad @\newline
            `` \quad '' \quad ``\,` '\,''\newline
            X-ray, 1--2, APP---Application\newline
            Good: One, two, Three\dots\newline
            Bad: One, two, Three...\newline
            One, two, Three\dots.
        
        \subsection{Special Punctuation} 
            \# \quad \$ \quad \% \quad \&\newline
            \{ \quad \} \quad \_ \quad \textbackslash\newline
            \~{} \quad \^{} \quad + | < > =

        \subsection{Other symbol}
            \S \quad \dag \quad \ddag \quad \P\newline
            \copyright \quad \textregistered \quad \texttrademark \quad \pounds \quad \textbullet\newline
            \texteuro \quad \textperthousand\newline
            \symbol{90} \quad \symbol{"5A} \quad \symbol{'132} \quad \symbol{`Z}

    \section{Blank}
        A B  C   D    E\newline
        \TeX ing and \TeX\ ing and \TeX{} ing and {\TeX} ing. \newline 
        Question~1\newline          % 1. Name and its sort
        Donald~E Knuth\newline      % 2. Given name
        Mr.~Knuth\newline           % 3. Acronym of name
        function~$f(x)$\newline     % 4. Short equation after its name
        1,~2, and 3\newline         % 5. Sequence
        
        A sentence. Another Sentence. \newline              % 1
        U.S.A. means United States of America. \newline     % 2
        Roman number XII\@. Yes. \newline                   % 3
        Thinker et al.\ made the double play. \\              % 4
        
        There is a phantom in the box. \\
        There is a \phantom{phantom} in the box. 
        % The command \phantom will generating a blank box
            % as wide as the paremeter. 

    \section{Wrap Around}
        This is a line. 

        This is another line. \newline
        This is a line. \newline
        This is another line. \linebreak
        This is another line. \linebreak[4]
        This is another very very very very 
        very very very very very lone line. \linebreak[3]
        This is another line. 

        
    \section{Font}
        \subsection{Preinstall Font}
            % font family
            \textrm{Roman font}\newline
            {\rmfamily Roman font}\newline              % the same as \textrm{}
            {\sffamily Sanserif font}\newline           % the same as \textsf{}
            {\ttfamily Typewriter font}\newline         % the same as \texttt{}
            % font shape
            {\upshape Upright shape}\newline            % the same as \textup{}
            {\itshape Italic shape}\newline             % the same as \textit{}
            {\slshape Slanted shape}\newline            % the same as \textsl{}
            {\scshape Small capital shape}\newline      % the same as \textsc{}
            % font series
            {\mdseries Medium series}\newline           % the same as \textmd{}
            {\bfseries Bold extended series}\newline    % the same as \textbf{}
            
            % mixed font
            {\rmfamily\slshape\mdseries Mixed font I}\newline
            \textsf\textit\textbf{Mixed font II}\newline    % NOTICE: The Italic shape did not work. 
                                                            % This is because every situation can be regarded as one kind of font. 
                                                            % And in this coordinates, the font is lack (and may have been replaced by other font).   
            {\itshape M}M \quad \textit{M}M \quad {\itshape M\/}M\newline
            \textit{M\nocorr}M\newline         % Forbid the revise.  
            `{\bfseries leaf}' \quad `{\bfseries leaf\/}' \quad `\textbf{leaf}'\newline
            {\sffamily \textbf{There is a \textit{complex example, 
            but \textnormal{a normal one} is in} it}}\newline
                % {\normalfont ...} is equal to it.  
            {\fontencoding{OT1}\fontfamily{pzc}\fontseries{m}\fontshape{n}\fontsize{14}{17}
            \selectfont PostScrip New Century Schoolbook}\newline
            {\usefont{T1}{pbk}{db}{n} PostScrip Bookman Demibold Normal. }
                

        \subsection{More Font}
            % {\inkfamily Hello!}
                
        \subsection{Math Font}
            \begin{equation}
                AB^2 = BC^2 + AC^2. 
            \end{equation}
            \begin{equation}
                \mathrm{AB^2 = BC^2 + AC^2. }
            \end{equation}

        \subsection{Sympol Font}
            % NOTE: The situation in this file, that babel package conflict with xelatex are very similar. 
            % symbol table with font's NFSS. 
            {\xfonttable{OT1}{lmr}{m}{n}}
            
    \section{Emphasize}
        This word is \emph{emphsized} in the sentence. \newline
        \textit{This word is \emph{emphsized} in the sentence.}\newline  
        This word is \Emph{emphsized} differently in the sentence. \newline
        This is an \underline{Emphasized} \underline{word}. \newline 
        This is an \uline{Emphasized} \uline{word}. \newline
        This is a very \uline{very very very very very very very very very very 
        very very very veryvery very very very} long sentence. 

    \section{Font Size} 
        {
            \fontsize{24.88pt}{36pt} \selectfont 
            A very huge\newline
            line
        }\newline  
        The text can be {\Large large}. \newline
        {\tiny tiny}\newline
        {\scriptsize scriptsize}\newline
        {\footnotesize footnotesize}\newline
        {\small small}\newline
        \ldots\newline
        {\LARGE large}\newline
        {\huge huge}\newline
        {\Huge Huge}\newline 

    \section{Line Space} 
        {
            \linespread{1.5} \selectfont
            A different\newline
            line space
        }\newline
        -----------\newline
        {\setstretch{1.5} A different\newline
        line space}\newline
        -----------\newline
        {\onehalfspacing A different\newline
        line space}\newline
        -----------\newline
        {\doublespacing A different\newline
        line space}\newline
        -----------\newline
        \begin{spacing}{1.5}
            A different\newline
            line space
        \end{spacing}
        -----------\newline
        \begin{doublespace}
            A different\newline
            line space
        \end{doublespace} 

    \section{Horizontal Space}
        % Underneath commands are not allowed to wrap around. 
        A\thinspace B or A\, B\newline
        C\negthinspace D\newline
        E\enspace F\newline
        G\nobreakspace H\newline or G~H
        % Underneath commands are allowed to wrap around. 
        A\quad B\newline
        C\qquad D\newline
        E\enskip F\newline
        G\ H\newline 
        Space\hspace{1cm}1\, cm

        \hspace{1cm} 1\, cm\newline
        \hspace*{1cm} 1\, cm
        
        \newcommand\test{longggggggggg\hspace{2em plus 1em minus 0.25em}}
        \test\test\test\test\test\test\test\test

        left\hspace{\fill}middle\hfill right
          
        left\hspace{\stretch{2}} middle\hspace{\stretch{3}} right
          
        left\dotfill middle\hrulefill right
        
        {
            \setlength\parindent{24em}
            Paragraph indent can be very wide in \LaTeX . 
        }\par
        % FIXME: Why this command above did not work?
        {
            Para\par
            \addtolength\parindent{2em}Para\par
            \addtolength\parindent{2em}Para
        }\par
        % There is a useful technique that is to set a definition of a designated length, 
            % so that you can just change your definition to change the global length. 
            % The mylen is defined in Introduction. 
        {
            Para\par
            \addtolength\parindent{\mylen}Para\par
            \addtolength\parindent{\mylen}Para
        }

    \section{Box}
        % TODO: box graphic 
        \mbox{Text in box cannot be broken. }
         
        This is a\,\makebox[1em]{text}\,in box. \par
        \makebox[5cm][s]{Some stretched text}

        \makebox[0pt][l]{word}text
         
        Some text\llap{word}\par
        \rlap{word}Some text
        
        \fbox{framed}\par
        \framebox[3cm][s]{A framed box}
         
        {
            \setlength\fboxsep{0pt}
            \setlength\fboxrule{0.1pt}
            \fbox{tight}
        }\par
        {
            \setlength\fboxsep{1em}
            \setlength\fboxrule{1pt}
            \fbox{loose}
        }
        
        \usebox\mybox \fbox{\usebox\mybox}
        
        \framebox[\wd\mybox]{text}
        
        \framebox[2\width]{frame}


\end{document}