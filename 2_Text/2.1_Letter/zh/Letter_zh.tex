%-*- coding: UTF-8 -*-
% Letter.tex
% 文字与符号

% 因为中文和西文环境不同,很多命令在中文环境下会出现不可预知的错误。
% 请先参阅 "./ENG/Letter_en.tex"

\documentclass[UTF8]{ctexart}
\usepackage{CJKfntef}

% -------- 使用 xelatex 编译 ---------
%\setCJKmainfont{Source Han Serif CN}            % 思源宋体
%\setCJKsansfont{Source Han Sans CN}             % 思源黑体
%\setCJKmonofont{FZQingKeBenYueSongJF}           % 方正清刻

%\setCJKfamilyfont{pen}{LiDeBiao-Xing3}          % 徳彪钢笔行书
% -------- 使用 xelatex 编译 ---------

\title{文字与符号}
\author{闻人沐妍}
\date{\today}

%\newtheorem{}[]

\begin{document}
    \maketitle

    \begin{abstract}
        文字与符号的编辑
    \end{abstract}

    \tableofcontents

    \section{字母}
        \subsection{重音}

        \subsection{其他字母}
        
        \subsection{连字}

    \section{标点符号}
        \subsection{普通标点}

        \subsection{特殊标点}

        \subsection{特殊符号}

    \section{空格}
        \mbox{条目}-a 和条目-b。
        % 为了避免中文和西文之间默认的空格,可以将中文装入一个盒子中。
        % 也可以使用 \CJKsetecglue{} 空白参数,表示中文与西文之间的内容为空。

    \section{换行}
        这是第一行\linebreak[4]
        另一行\linebreak
        另一行\linebreak[2]
        另一行\linebreak[0]
        最后一行

    \section{字体}
        \subsection{预设字体}
            % 中文没有西文那么多的变体,许多字体都是独立的。
            % 一般只使用不同的字体族区分中文。
            % 在 xeCJK 宏包下,中文和西文字体的选择命令是分开的。
            {\CJKfamily{zhhei}这是黑体}\newline
            {\CJKfamily{zhkai}这是楷体}\newline
            % 不同的系统下,字体族可能会有差异。此外还有 zhsong, zhfs。
            % xeCJK 宏包还预定义了一些字体命令。
            {\songti 宋体} \quad {\heiti 黑体}\newline
            {\fangsong 仿宋} \quad {\kaishu 楷书}\newline
            % ctex 宏包还定义了一套组合的字体,使文章中实现像西文一样的粗体、意大利体等效果。
            % 比如 Windows 系统下,rm 是宋体,bf 是黑体,it 是楷体,sf 是幼圆,tt 是仿宋。
            {\CJKfamily{rm}这是宋体}\newline
            \textbf{粗体} \quad \textit{斜体}       % 等同于 \bfseries, \itshape 命令。
            % 使用其他字体时,则可能出现由于找不到字体的情况出错。
            % NOTE: 这部分字体设置可在下面文件中更改
            % ".../texlive/2021/texmf-dist/tex/latex/ctex/fontset/ctex-fontset-windows.def"
            % 在文件中,ctex 宏包通过使用下面的命令为某个字体的变体指定代替的其他字体
            % \setCJKmainfont[BoldFont=SimHei, ItalicFont=KaiTi, BoldItalicFont=LiSu]{SimSun}
            % 和西文中 \setmainfont 类似。
            % 使用 \renewcommand\CJKrmdefault{font} 可以将全文使用的 rm 字体族替换为你制定的字体族。
            % 使用 \renewcommand\CJKfamilydefault{\CJKsfdefault},可以将全文默认的字体族修改为 sf 字体族。

        \subsection{更多字体}
            % 使用下面的命令修改预设字体
            % \setCJKmianfont[optiions]{font1}         % rm
            % \setCJKsansfont[optiions]{font2}         % sf
            % \setCJKmonofont[optiions]{font3}         % tt
            %\CJKfamily{tt}{印刷字体}\newline
            % 5.1 最后给了一个 option 的实例。
            % xeCJK 宏包还给出了一个自定义字体族的命令。
            %\setCJKfamilyfont{defined_name}[options]{font4}
            % 使用这个命令后,就可以使用你定义的字体族名称调用字体了。
            %\CJKfamily{pen}{钢笔手写体}
        
    \section{强调}
        % --------------- 使用 xelatex 编译 ----------------
        %\CJKunderdot{下加点}\newline
        %\CJKunderline{下划线}\newline
        %\CJKunderdblline{双下划线}\newline
        %\CJKunderwave{波浪线}\newline
        %\CJKsout{横删除线}\newline
        %\CJKxout{斜删除线}\newline
        \begin{CJKfilltwosides}{5cm}
            分散对齐。
        \end{CJKfilltwosides}\newline
        \emph{强调}
        % "https://www.ctan.org/tex-archive/languages/chinese/CJK/cjk-4.8.4/texinput/#CJKfntef.sty"
        % --------------- 使用 xelatex 编译 ----------------
        % 使用 CJKfntef 宏包后,原来的命令 \emph 如果被改为下划线格式,同样可以使用 \normalem 改回原来的定义。
        % 在 ctex 宏包或文档类中,也可以使用 fntef 选项调用 CJKfntef

    \section{字号}
        % 在 LaTeX 中,字号有 0~8 和 -0~-6 这几类,分别代表 初号、一号到八号和小初号、小一号到小六号。
        {\zihao{0} 初号}\newline
        {\zihao{-0} 小初号}\newline
        {\zihao{1} 一号}\newline
        {\zihao{-1} 小一号}\newline
        {\zihao{4} 四号}\newline
        {\zihao{8} 八号}\newline
        % ctex 的文档类提供了两个默认字体大小的选项,c5size 和 cs4size,

\end{document}

