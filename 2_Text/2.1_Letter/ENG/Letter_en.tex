%-*- coding: UTF-8 -*-
% Letter.tex
% Letetr and Symbol

% ---------------------- Introduction ----------------------
\documentclass{article}

\usepackage[OT2,OT1,T1]{fontenc}        % text encoding selection, may have problem in lualatex
\usepackage[greek,english]{babel}       % language, may have problem in lualatex. TODO: See the babel documentation. 
\usepackage[utf8]{inputenc}             % FIXME: input encoding?
\usepackage{textcomp}                   % provide special symbol command
\usepackage{fonttable}
\usepackage[normalem]{ulem}
\usepackage{setspace}
\newcommand\Emph{\textbf}
%\usepackage{txfonts}                   % may conflict with fontspec
%\usepackage{fontspec}

%\usepackage{ccfonts}
%\usepackage[no-math]{fontspec}

%\setmainfont{Garamond}                 % depend on fontspec, and do not contain greek letter and cyrillic. 
%\setsansfont{Arial}                    % depend on fontspec
%\setmonofont{Edwardian Script ITC}     % depend on fontspec

%\newfontfamily\inkfamily{Ink Free}     % set new command inkfamily to use Ink Free, only support by fontspec package. 
%\renewcommand\rmdefault{cmfib}         % change the rm family into cmfib

\setlength\lineskiplimit{2.5bp}         % lineskiplimit
\setlength\lineskip{2.5bp}              % lineskip

\newlength\mylen                        % define mylen
\setlength\mylen{2em}

\newsavebox\mybox                       % define mybox
\sbox\mybox{text text}

\title{Letter and Symbol}
\author{Wenren Muyan}
\date{\today}

% ---------------------- Introduction ----------------------

\begin{document}
    \maketitle

    \begin{abstract}
        Input letter and symbol. 
    \end{abstract}

    \tableofcontents

    \section{Letter}
        \subsection{Accents}
            \`o\phantom{50}\'o\phantom{50}\^o\phantom{50}\"o\newline
            \~o\phantom{50}\=o\phantom{50}\.o\phantom{50}\u{o}\newline
            \v{o}\phantom{50}\H{o}\phantom{50}\t{oo}\phantom{50}\r{o}\newline
            \c{o}\phantom{50}\d{o}\phantom{50}\b{o}\newline
            \AA\phantom{50}\aa\phantom{50}\AE\phantom{50}\ae\newline
            \OE\phantom{50}\oe\phantom{50}\SS\phantom{50}\ss\newline
            \IJ\phantom{50}\ij\phantom{50}\L\phantom{50}\l\newline
            \O\phantom{50}\o\phantom{50}\i\phantom{50}\j
            % Some example of accents on letters and some LaTeX letters. 

        \subsection{Other Letters}
            \textgreek{abcde}\newline                   % would produce error in lualatex
            %\textrussian{abcde}
            % More details in https://ctan.math.utah.edu/ctan/tex-archive/macros/latex/contrib/babel-contrib/greek/usage.pdf
            {\fontencoding{OT2}\selectfont ABCDE}       % would produce error in lualatex
            % Input greek letter and cyrillic by using babel or fontenc macros. 
            % More details in http://latex3.github.io/help/documentation/cyrguide.pdf
            % NOTICE: Some letters need support all of the editor, input font and output font. 
            % TODO: More details of using babel are in http://www.leliseron.org/annexes/LaTeX/babel.pdf
            % FIXME: How to input Russian with babel? 
            % REVIEW: 
            % DEBUG:
            % NOTE:

        \subsection{Ligature}
            differ find flight difficult ruffle\newline
            dif{}fer f{}ind f\/light dif\/f\/icult ruffle
            % To avoid ligature betweent tow letters, 
            % There are tow ways taht you can do like above. 

    \section{Punctuation}
        \subsection{Common Punctuation}
            , \quad . \quad ; \quad :\newline
            ! \quad ? \quad ` \quad '\newline
            ( \quad ) \quad [ \quad ]\newline
            \quad - \quad / \quad * \quad @\newline
            % DCB case punctuation that can input directly
            % SBC case punctuation is similar to DCB, but only supported in Chinese environment. 
            `` \quad '' \quad ``\,` '\,''\newline
            % Use \, to split " and ' if they are close to each other. 
            X-ray, 1--2, APP---Application\newline
            % Different number of '-' will generate different puncatuation. 
            % NOTICE: The habit of using them in English writing. 
            Good: One, two, Three\dots\newline
            Bad: One, two, Three...\newline
            % The Command \dots will generate an ellipsis 
                % with wider distance between dots which has a better looking
                % than directly using three dots punctuation. 
            % Command \dots is equal to \ldots. 
            One, two, Three\dots.
            % Another dots that used to be full stop could follow \dots. 
        
        \subsection{Special Punctuation}
            % Some symbol or punctuation are occupied by LaTeX command
                % which can not input directly 
                % and need to input specific command instead. 
                % We call the "\" escape character. 
            \# \quad \$ \quad \% \quad \&\newline
            \{ \quad \} \quad \_ \quad \textbackslash\newline
            \~{} \quad \^{} \quad 
            % TODO: Input |, <, >, +, =

        \subsection{Other symbol}
            \S \quad \dag \quad \ddag \quad \P\newline
            \copyright \quad \textregistered \quad \texttrademark \quad \pounds \quad \textbullet\newline
            % Above is the special symbol frequently using in an artical
                % in latex default environment. 
                % You can also type the symbol through UTF-8 coding. 
            % NOTICE: Other symbol have dependence on different font. 
            % Latex basic macro package textcomp has designed many symbol would appear in a text. 
            \texteuro \quad \textperthousand\newline
            % Package tipa provide symbol of phonetic, 
                % dingbat, bbding, pifont provide other symbol of instruction and decoration. 
            % TODO: Look up the package above how to use. 
            % See "./symbols-letters.pdf" for more symbols. 
            % NOTICE: Some package just provide symbol command, 
                % some package provide alternative font to their symbol, 
                % while some package change the holistic font set of the artical. 
                % So please read related documentation of your package before using. 
            \symbol{90} \quad \symbol{"5A} \quad \symbol{'132} \quad \symbol{`Z}
            % Command \symbol allow you type the symbol's coding to repalace it. 
            % The coding can be decimal, octonary, hexadecimal and the symbol its self. 
                % The format example is given above. 
                % NOTICE: If the symbol is special punctuation in 2.2, an escape character is also needed in front. 
                % NOTICE: Chinese character is only allowed in xelatex environment.

    \section{Blank}
        A B  C   D    E\newline
        % NOTICE: A series of space is equal to a click of space. 
        \TeX ing and \TeX\ ing and \TeX{} ing and {\TeX} ing. \newline
        % The space after a macro command will be ignored. 
        % 4 ways to avoid the ignoring situation avove. 
        Question~1\newline          % 1. Name and its sort
        Donald~E Knuth\newline      % 2. Given name
        Mr.~Knuth\newline           % 3. Acronym of name
        function~$f(x)$\newline     % 4. Short equation after its name
        1,~2, and 3\newline         % 5. Sequence
        % These situation is not allowed to interrupt two words. 
        % However, blank generating by click the space key could be interrupted
            % if there is a line feed or something else. 
        % Using ~ to link two words can prevent them to be interrupted. 
        A sentence. Another Sentence. \newline              % 1
        U.S.A. means United States of America. \newline     % 2
        Roman number XII\@. Yes. \newline                   % 3
        Thinker et al.\ made the double play.               % 4
        % In English writing, the space must be used after ", . ; :". 
            % TODO: See more details of rules in English writing in book--
            % Robert Bringhurst's The Element of Typographic Style, 
            % Anonymous' The Manual of Style. 
        % In LaTeX, the width of the space at the end of a sentence following a dot
            % is different from in the middle. 
        % But sometimes, a dot could be used 
            % in the middle of a sentence following a samll letter like situation 4. 
            % or at the end of a sentence following a capital letter like situation 3. 
        % Use command \@. in the situation 3 and use '\' before a sapce in the situation 4.  
        % While you compiling Chinese documentation with xelatex, 
            % a space while automatially add between Chinese and English
            % ignoring how many space you have typed in these space. 
        % Also there are some methods to prevent the space from generating by the compiler. 
        % Please see Chinese documentation for more infomation. 
        There is a phantom in the box. 
        There is a \phantom{phantom} in the box. 
        % The command \phantom will generating a blank box
            % as wide as the paremeter. 

    \section{Line Feed}
        This is a line. 

        This is another line. \newline
        % In latex, just one line feed would be ignored, 
            % this peculiarity is set to help the code more legible. 
        % To begin a new paragraph, you need two line feed
            % with a blank line in the middle. 
        This is a line. \newline
        This is another line. 
        % Another way to begin a new line is to use command
            % \\ or \newline at the end of a line. 
        % Command \linebreak will distribute the former from left to right margin in Chinese environment. 
        % NOTICE: Both \\ and \linebreak have hbox problem in common environment, 
            % they are usually used in poetry, table and math. 
        % Command \\ have a parameter which set the vertical distance between two lines. 
            % If the line after \\ begin with "[", you need using {} after \\ with blank content. 
        % And \linebreak have a command which set the degree of line feed and be defaultly set to 4, 
        % ranging from 0 the lowest to 4 the highest being contrary to the command \nolinebreak. 
        
    \section{Font}
        \subsection{Preinstall Font}
            % font family
            \textrm{Roman font}\newline
            {\rmfamily Roman font}\newline              % the same as \textrm{}
            {\sffamily Sanserif font}\newline           % the same as \textsf{}
            {\ttfamily Typewriter font}\newline         % the same as \texttt{}
            % font shape
            {\upshape Upright shape}\newline            % the same as \textup{}
            {\itshape Italic shape}\newline             % the same as \textit{}
            {\slshape Slanted shape}\newline            % the same as \textsl{}
            {\scshape Small capital shape}\newline      % the same as \textsc{}
            % font series
            {\mdseries Medium series}\newline           % the same as \textmd{}
            {\bfseries Bold extended series}\newline    % the same as \textbf{}
            % The font can be divided into three orthometric dimensions--family, shape and series. 
            % There are two ways to change the font which are font declaration as the examples above. 
                % and box command with parameter as the explaination in annotation. 
            % In latex, available options are given above. 
                % Since some font may not have so all shape or series, 
                % these two command may have different effect from usual font. 
            % Chinese fonts usaully have no other variant, so 
                % different type of fonts are used to replace the variant in a font family. 
                % See more details in Chinese Documentation. 
            % mixed font
            {\rmfamily\slshape\mdseries Mixed font I}\newline
            \textsf\textit\textbf{Mixed font II}\newline    % NOTICE: The Italic shape did not work. 
                                                            % This is because every situation can be regarded as one kind of font. 
                                                            % And in this coordinates, the font is lack (and may have been replaced by other font). 
            % The font's three dimensions can be mixed optionally. 
            % The last slanted letter at the end of a line will cross the right margin. 
                % Use the methods underneath to revise it. 
            {\itshape M}M \quad \textit{M}M \quad {\itshape M\/}M\newline
            \textit{M\nocorr}M\newline         % Forbid the revise. 
            % Also, bold letters may be too close to the puntuation following it. 
                % Underneath are methods to solve it. 
            `{\bfseries leaf}' \quad `{\bfseries leaf\/}' \quad `\textbf{leaf}'\newline
            % EXTENSION: Command \newcommand could be used to reset \nocorrlist{}. 
                % \nocorrlist{} is defualt set with "," and "." in it 
                % which is the same as \newcommand\nocorrlist{,.}
            % If you want to change back to default font in a complex font environment, 
                % do like underneath
                \newline
            {\sffamily \textbf{There is a \textit{complex example, 
            but \textnormal{a normal one} is in} it}}\newline
                % {\normalfont ...} is equal to it.  
                % And the font you will get using this command is \rmfamily\mdseries\upshape. 
            % EXTENSION: More details about English font can be found in "./hartke.pdf" and https://tug.org/FontCatalogue/
            % Like our dividing font into three dimensions, latex publish its own methods to locate a font, 
                % NFSS(New Font Selection Scheme) which also dividing font into many orthometric properties, 
                % such as encoding, family, series, shape, size and so on. 
            {\fontencoding{OT1}\fontfamily{pzc}\fontseries{m}\fontshape{n}\fontsize{14}{17}
            \selectfont PostScrip New Century Schoolbook}\newline
            {\usefont{T1}{pbk}{db}{n} PostScrip Bookman Demibold Normal. }
                % The first parameter in fontsize is letter's size, and the second one is line spacing. 
            % We can also replace the rmfamily font by command \renewcommand\rmdefault{cmfib}, 
            % so that we will use fontfamily cmfib when we use \rmfamily or \textrm{}. 
            % Also we can change the artical's using default font family from rm into sf, 
            % using command \renewcommand\familydefault{\sfdefault}. 

        \subsection{More Font}
            % In 5.1, those command only can use the fonts that are preinstalled in System or latex. 
            % Sometimes, we lack some fonts of them or want to use other fonts. 
            % Some packages can replace the font configuration, such as times, mathptms, txfonts. 
            % NOTICE: Different font may need different encoding, 
                % For example, if you want to use font package ccfonts and euler 
                % for better harmony of text and math part, you need to use T1 encoding. 
                % Use command NOTE: \usepackage[encoding1, encoding2, ... encodingn]{fontenc}
                    % The parameters are the encoding ways of the article, with the final one the default one. 
            % The package fontspec contains commands that can use other font to replace the default font in latex. 
                % \usepackage{fontspec}
                % \setmainfont{font1}       % change the \rmfamily
                % \setsansfont{font2}       % change the \sffamily
                % \setmonofont{font3}       % change the \ttfamily
            % See code above in preamble. 
            % Fontspec also can define new font famlily if needed. 
                % \newfontfamily\family_name{font}
            %{\inkfamily Hello!}

        \subsection{Math Font}

            % Generally LaTeX using italic font in math parts of the article. 
            % And command /mathrm{}, /mathsf{}, /mathtt{}, and so on to change the math parts font, 
            % which is configured defualtly as the same with text. 
            % Some math font package can change the math parts font, but whether it is useful to above command 
            % depanding the package whether loads "no-math" module in fontspec. 
            % If it doesn't like package ccfonts, and you want to change the configuration of these command like \mathrm{}, 
            % NOTE: just manually load the module by command \usepackage[no-math]{fontspec}
            \begin{equation}
                AB^2 = BC^2 + AC^2. 
            \end{equation}
            \begin{equation}
                \mathrm{AB^2 = BC^2 + AC^2. }
            \end{equation}

            % TODO: Situation may go complex when we use both English which not support Unicode and Chinese in one article, 
            % See the book in Page 74 for some help. 

        \subsection{Sympol Font}
            % LaTeX only use one encoding in one time. 
            % NOTICE: If you need different encodings, you need change it manually. 
            % There is a example, in Greek and fontspec package. 
                % XeLaTeX use Unicode encoding, while the command \textipa in tipa package which outputs IPA works in ASCII. 
                % So when you compile command \textipa{}, the compiler will change the IPA's encoding from ASCII into Unicode, 
                % which can not output by default font Latin Modern because it less these endoing. 
                % NOTE: The situation in this file, that babel package conflict with xelatex are very similar. 
                % To solve this problem, you just need use font containing IPA in Unicode, 
                % such as Linux Libertine O, Times New Roman and so on, for example, NOTE: \setmainfont{CMU Serif}. 
                % TODO: There is another way is use \newcommand to define a new command 
                % referencing the \textipa command's defination with encoding changing options. 
                % Since I am not still understand the \newcommand, please see more details in book Page 75. 
            % There is a useful package which can output the symbol table in specific font with its NFSS. 
            {\xfonttable{OT1}{lmr}{m}{n}}
            
    \section{Emphasize}
        % In LaTeX, basical command to emphasize is actually to use italic in upright or on the contrary. 
        This word is \emph{emphsized} in the sentence. \newline
        \textit{This word is \emph{emphsized} in the sentence.}\newline 
        % So you can use \newcommand to define your own style command to emphasize your text. 
        This word is \emph{emphsized} differently in the sentence. \newline
        % Another way to emphsize is making underline. 
        This is an \underline{Emphasized} \underline{word}. \newline
        % Howerver, the underline command does not support wraping around text, and not align under the text. 
        % Package ulem provide a better command. 
        This is an \uline{Emphasized} \uline{word}. \newline
        This is a very \uline{very very very very very very very very very very 
        very very very veryvery very very very} long sentence. 
        % NOTICE: The ulem package will change the original \emph command into underline style. 
        % To avoid this, we can use normalem option when we use ulem package, or just use \normalem to replace it. 
        % Also, the ulem package support the colorful highlight. 
        % % TODO: It's a good idea to see more usage of ulem package in "../ulem.pdf". 
        % CJK package provide its own command to emphaisze which is more friendly in Chinese Situation. 

    \section{Font Size}
        % In LaTeX, there are 11 kinds of units to represent the length. 
        % There are pt(point), pc(pica), in(inch), bp(big point), cm(centimeter), mm(millimeter), dd(didot point), 
            % sp(scaled point), em(equal to the font size using and \quad command's length), ex(x-height, defined by font itsself)
        % At past, font size was regarded as one of the coordinates in NFSS. 
            % But now, fonts are usually vectors, which can change its size at will. 
            % Font size is described by length units in LaTeX. 
        % Command \fontsize{}{} is used to select font size specificly. 
        % The first parameter defines the font size and the second parameter defines the line space which would be explained in section 8. 
        {
            \fontsize{24.88pt}{36pt} \selectfont 
            A very huge\newline
            line
        }\newline
        % In LaTeX, there are some simple declaration command provided to change the font size easily. 
        The text can be {\Large large}. \newline
        % They, sorting from the smallest to the largest, are
            % \tiny, \scriptsize, \footnotesize, \small, \normalsize, \large, \Large, \LARGE, \huge, \Huge
        {\tiny tiny}\newline
        {\scriptsize scriptsize}\newline
        {\footnotesize footnotesize}\newline
        {\small small}\newline
        \ldots\newline
        {\LARGE large}\newline
        {\huge huge}\newline
        {\Huge Huge}\newline
        % In different document class options, the \normalsize could be configured differently. 
        % Standard class have 10pt, 11pt, 12pt normalsize to option with command above defineing different size of font. 
        % And command above will change the line space, in order to make the composing more beautiful.
        % NOTICE: Command \quad is as wide as the font size's length, in other words, as wide as the letter's height. 
         
        % In Chinese environment, it is more intuitively to use \zihao to represent \fontsize instead. 
            % And zihao is a special font size units for chinese characters. See more details in Chinese documentation. 

    \section{Line Space}
        % TODO: line space and font size graph
        % The adjacent two lines have their base line, and the distance between two base lines is line space. 
        % Defaultly, line space in paragraph is 1.2 times as long as the font size. 
        % The basical command can be used like underneath, the parameters defines the times to default line space. 
        {
            \linespread{1.5} \selectfont
            A different\newline
            line space
        }\newline
        -----------\newline
        % In Chinese environment, the default line space is 1.3 times as long as the font size. 
        % Package setspace provides some command to define space line easily. 
        % Among them, the command \setstretch{} which is used frequently is equal to \linespread{}\selectfont
        {\setstretch{1.5} A different\newline
        line space}\newline
        -----------\newline
            % NOTICE: It should be noticed that other command like \singlespacing, \onehalfspacing 
            % reprensent the line space times to fontsize, instead of the default space line. 
        {\onehalfspacing A different\newline
        line space}\newline
        -----------\newline
        {\doublespacing A different\newline
        line space}\newline
        -----------\newline
            % These commands can also used as environment, for example
        \begin{spacing}{1.5}
            A different\newline
            line space
        \end{spacing}
        -----------\newline
        \begin{doublespace}
            A different\newline
            line space
        \end{doublespace}
        % In LaTeX, the line space is controlled by the value of \baselineskip. 
        % Actually, \fontsize and \linespread indirectly controlled the \baselineskip. 
        % lineskiplimit is a limit value set by us whose function is that when two adjancent lines' line space
            % smaller than it, the line space will use the value \lineskip, also set by us, instead. 
            % The example is placed in introduction part. 
            % This function can be very useful when math fraction appears in your article. 

    \section{Horizontal Space}
        % Line Space is the vertical distance between line and line, 
        % while horizontal space is between letter and letter in the same line. 
        % Some simple command can be used in text. 
        % Underneath commands are not allowed to wrap around. 
        A\thinspace B or A\, B\newline
        C\negthinspace D\newline
        E\enspace F\newline
        G\nobreakspace H\newline or G~H
        % Underneath commands are allowed to wrap around. 
        A\quad B\newline
        C\qquad D\newline
        E\enskip F\newline
        G\ H\newline
        % Command \hspace{} can adjust the horizontal distance precisely through the length parameter. 
            % \hspace{} is allowed to wrap around. 
        Space\hspace{1cm}1\, cm\newline
        % However, \hspace{} will ignore the distance when its left has no text. 
            % In this situation, we can use \hspace*{} instead. 
        \hspace{1cm} 1\, cm\newline
        \hspace*{1cm} 1\, cm\newline
        % \hspace also can generate a length ranging in a designated scale which commonly called glue or rubber length. 
            % FIXME: I do not understand how it works. 
        \newcommand\test{longggggggggg\hspace{2em plus 1em minus 0.25em}}
        \test\test\test\test\test\test\test\test\newline
        % \fill is a special rubber length parameter in \hspace, which ranging from zero to infinite. 
            % The command \hspace{\fill} can be written \hfill as its abbreviation. 
        left\hspace{\fill}middle\hfill right\newline
        % Parameter \stretch{} can generate a rubber length which elastic force is parameter times as strong as the \hfill. 
        left\hspace{\stretch{2}} middle\hspace{\stretch{3}} right\newline
        % Commands \dotfill and \hrulefill is similar to \hfill in function, 
        % whose differences are that the former uses dot and the latter uses underline replacing the blank. 
        left\dotfill middle\hrulefill right\newline
        % Command \parindent can control the indent at the far left of one line. 
            % It usually used in environment and need to coordinate with \setlength. See the example underneath
        {
            \setlength\parindent{24em}
            Paragraph indent can be very wide in \LaTeX . 
        }\par
        % FIXME: Why this command above did not work?
        {
            Para\par
            \addtolength\parindent{2em}Para\par
            \addtolength\parindent{2em}Para
        }\par
        % There is a useful technique that is to set a definition of a designated length, 
            % so that you can just change your definition to change the global length. 
            % The mylen is defined in Introduction. 
        {
            Para\par
            \addtolength\parindent{\mylen}Para\par
            \addtolength\parindent{\mylen}Para
        }

    \section{Box}
        % TODO: box graphic
        % REVIEW:  Box is the fundamental unit LaTeX is dealing with. 
        % Everything, containing a letter, a line, a paragraph, a graph, a table, a page and so on, 
            % can be regarded as a box. 
        % On horizon, the letter boxes constitude a line box, and on vertical, the line boxes contitude a page box. 
        % Command \mbox{} is the most simple command which will generate a box with text put into. 
            % The text in it is the same as outer text, but the blank before or after it could be affected in some special situation, 
            % and most commonly, it do not allowed wrap around that is the function every box does. 
        \mbox{Text in box cannot be broken. }\par
        % Command \makebox[][]{text} has three parameters. 
            % The first one represents the width, 
            % The second one represents the alignment, with c-center, l-left, r-right, s-stretch optionally. 
        This is a\,\makebox[1em]{text}\,in box. \par
        \makebox[5cm][s]{Some stretched text}\par
        % The \makebox can even generate a 0-width box to make overlap effect. 
        \makebox[0pt][l]{word}text\par
        % EXTENSION: The Command \llap and \rlap is professionally used to make overlap effect, 
            % the former one left alignment and the latter one right alignment. 
        Some text\llap{word}\par
        \rlap{word}Some text\par
        % Command \fbox and \framebox can generate box with framwork. Their grammer is similar to \mbox and \makebox. 
        \fbox{framed}\par
        \framebox[3cm][s]{A framed box}\par
        % The space between framework and text is configured by \fboxsep which is defaultly 3pt. 
        % And the thickness of the framework is configured by \fboxrule which is defaultly 0.4pt. 
        {
            \setlength\fboxsep{0pt}
            \setlength\fboxrule{0.1pt}
            \fbox{tight}
        }\par
        {
            \setlength\fboxsep{1em}
            \setlength\fboxrule{1pt}
            \fbox{loose}
        }\par
        % Command \newsavebox can define a box environment so that you can use it repeatedly. 
            % The mybox is defined in Introduction. 
        \usebox\mybox \fbox{\usebox\mybox}\par
        % Saving box is usually used to saving some complex content. 
        % Command \settowidth<length_variable>, \settoheight<length>, \settodepth<length>{text} 
        % can respectively get the text parameter's width, height, and depth to the text. 
            % And command \wd<box_variable>, \ht<box>, \dp<box> can respectively get the box parameter's width, height, and depth to the text.
        \framebox[\wd\mybox]{text}\par
        % Some special parameter represent the text natural properties. 
            % \width, \height, \depth, \totalheight(the sum of height and depth)
        \framebox[2\width]{frame}


\end{document}