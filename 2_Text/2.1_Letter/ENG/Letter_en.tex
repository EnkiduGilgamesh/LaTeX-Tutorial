%-*- coding: UTF-8 -*-
% Letter.tex
% Letetr and Symbol

\documentclass{article}

\usepackage[OT2,T1]{fontenc}
\usepackage[greek,english]{babel}
\usepackage[utf8]{inputenc}
\usepackage{textcomp}

\title{Letter and Symbol}
\author{Wenren Muyan}
\date{\today}

\begin{document}
    \maketitle

        \begin{abstract}
        Input letter and symbol. 
    \end{abstract}

    \tableofcontents

    \section{Letter}
        \subsection{Accents}
            \`o\phantom{50}\'o\phantom{50}\^o\phantom{50}\"o\newline
            \~o\phantom{50}\=o\phantom{50}\.o\phantom{50}\u{o}\newline
            \v{o}\phantom{50}\H{o}\phantom{50}\t{oo}\phantom{50}\r{o}\newline
            \c{o}\phantom{50}\d{o}\phantom{50}\b{o}\newline
            \AA\phantom{50}\aa\phantom{50}\AE\phantom{50}\ae\newline
            \OE\phantom{50}\oe\phantom{50}\SS\phantom{50}\ss\newline
            \IJ\phantom{50}\ij\phantom{50}\L\phantom{50}\l\newline
            \O\phantom{50}\o\phantom{50}\i\phantom{50}\j
            % Some example of accents on letters and some LaTeX letters. 

        \subsection{Other Letters}
            \textgreek{abcde}\newline
            %\textrussian{abcde}
            {\fontencoding{OT2}\selectfont ABCDE}
            % Input greek letter and cyrillic by using babel or fontenc macros. 
            % NOTICE: Some letters need support all of the editor, input font and output font. 
            % TODO: More details of using babel are in http://www.leliseron.org/annexes/LaTeX/babel.pdf
            % FIXME: How to input Russian with babel?
            % REVIEW: 
            % DEBUG:
            % NOTE:

        \subsection{Ligature}
            differ find flight difficult ruffle\newline
            dif{}fer f{}ind f\/light dif\/f\/icult ruffle
            % To avoid ligature betweent tow letters, 
            % There are tow ways taht you can do like above. 

    \section{Punctuation}
        \subsection{Common Punctuation}
            , \quad . \quad ; \quad :\newline
            ! \quad ? \quad ` \quad '\newline
            ( \quad ) \quad [ \quad ]\newline
            \quad - \quad / \quad * \quad @\newline
            % DCB case punctuation that can input directly
            % SBC case punctuation is similar to DCB. 
            `` \quad '' \quad ``\,` '\,''\newline
            % Use \, to split " and ' if they are close to each other. 
            X-ray, 1--2, APP---Application\newline
            % Different number of '-' will generate different puncatuation. 
            % NOTICE: The habit of using them in English writing. 
            Good: One, two, Three\dots\newline
            Bad: One, two, Three...\newline
            % The Command \dots will generate an ellipsis 
                % with wider distance between dots which has a better looking
                % than directly using three dots punctuation. 
            % Command \dots is equal to \ldots. 
            One, two, Three\dots.
            % Another dots that used to be full stop could follow \dots. 
        
        \subsection{Special Punctuation}
            % Some symbol or punctuation are occupied by LaTeX command
                % which can not input directly 
                % and need to input specific command instead. 
                % We call the "\" escape character. 
            \# \quad \$ \quad \% \quad \&\newline
            \{ \quad \} \quad \_ \quad \textbackslash\newline
            \~{} \quad \^{} \quad 
            % TODO: Input |, <, >, +, =

        \subsection{Other symbol}
            \S \quad \dag \quad \ddag \quad \P\newline
            \copyright \quad \textregistered \quad \texttrademark \quad \pounds \quad \textbullet\newline
            % Above is the special symbol frequently using in an artical
                % in latex default environment. 
                % You can also type the symbol through UTF-8 coding. 
            % NOTICE: Other symbol have dependence on different font. 
            % Latex basic macro package textcomp has designed many symbol would appear in a text. 
            \texteuro \quad \textperthousand\newline
            % Package tipa provide symbol of phonetic, 
                % dingbat, bbding, pifont provide other symbol of instruction and decoration. 
            % TODO: Look up the package above how to use. 
            % See "./symbols-letters.pdf" for more symbols. 
            % NOTICE: Some package just provide symbol command, 
                % some package provide alternative font to their symbol, 
                % while some package change the holistic font set of the artical. 
                % So please read related documentation of your package before using. 
            \symbol{90} \quad \symbol{"5A} \quad \symbol{'132} \quad \symbol{`Z}
            % Command \symbol allow you type the symbol's coding to repalace it. 
            % The coding can be decimal, octonary, hexadecimal and the symbol its self. 
                % The format example is given above. 
                % NOTICE: If the symbol is special punctuation in 2.2, an escape character is also needed in front. 
                % NOTICE: Chinese character is only allowed in xelatex environment.

    \section{Blank}
        A B  C   D    E\newline
        % NOTICE: A series of space is equal to a click of space. 
        \TeX ing and \TeX\ ing and \TeX{} ing and {\TeX} ing. \newline
        % The space after a macro command will be ignored. 
        % 4 ways to avoid the ignoring situation are avove. 
        Question~1\newline          % 1. Name and its sort
        Donald~E Knuth\newline      % 2. Given name
        Mr.~Knuth\newline           % 3. Acronym of name
        function~$f(x)$\newline     % 4. Short equation after its name
        1,~2, and 3\newline         % 5. Sequence
        % These situation is not allowed to interrupt two words. 
        % However, blank generating by click the space key could be interrupted
            % if there is a line feed or something else. 
        % Using ~ to link two words can prevent them to be interrupted. 
        A sentence. Another Sentence. \newline              % 1
        U.S.A. means United States of America. \newline     % 2
        Roman number XII\@. Yes. \newline                   % 3
        Thinker et al.\ made the double play.       % 4
        % In English writing, the space must be used after ", . ; :". 
            % TODO: See more details of rules in English writing in book--
            % Robert Bringhurst's The Element of Typographic Style, 
            % Anonymous' The Manual of Style. 
        % In LaTeX, the width of the space at the end of a sentence following a dot
            % is different from in the middle. 
        % But sometimes, a dot could be used 
            % in the middle of a sentence following a samll letter like situation 4. 
            % or at the end of a sentence following a capital letter like situation 3. 
        % Use command \@. in the situation 3 and use '\' before a sapce in the situation 4.  
        % While you compiling Chinese documentation with xelatex, 
            % a space while automatially add between Chinese and English
            % ignoring how many space you have typed in these space. 
        % Also there are some methods to prevent the space from generating by the compiler. 
        % Please see Chinese documentation for more infomation. 
        There is a phantom in the box. 
        There is a \phantom{phantom} in the box. 
        % The command \phantom will generating a blank box
            % as wide as the paremeter. 

    \section{Line Feed}
        This is a line. 

        This is another line. \newline
        % In latex, just one line feed would be ignored, 
            % this peculiarity is set to help the code more legible. 
        % To begin a new paragraph, you need two line feed
            % with a blank line in the middle. 
        This is a line. \newline
        This is another line. 
        % Another way to begin a new line is to use command
            % \\ or \newline at the end of a line. 
        % Command \linebreak will distribute the former from left to right margin in Chinese environment. 
        % NOTICE: Both \\ and \linebreak have hbox problem in common environment, 
            % they are usually used in poetry, table and math. 
        % Command \\ have a parameter which set the vertical distance between two lines. 
            % If the line after \\ begin with "[", you need using {} after \\ with blank content. 
        % And \linebreak have a command which set the degree of line feed and be defaultly set to 4, 
        % ranging from 0 the lowest to 4 the highest being contrary to the command \nolinebreak. 
        
    \section{Font}
        % font family
        \textrm{Roman font}\newline
        {\rmfamily Roman font}\newline              % the same as \textrm{}
        {\sffamily Sanserif font}\newline           % the same as \textsf{}
        {\ttfamily Typewriter font}\newline         % the same as \texttt{}
        % font shape
        {\upshape Upright shape}\newline            % the same as \textup{}
        {\itshape Italic shape}\newline             % the same as \textit{}
        {\slshape Slanted shape}\newline            % the same as \textsl{}
        {\scshape Small capital shape}\newline      % the same as \textsc{}
        % font series
        {\mdseries Medium series}\newline           % the same as \textmd{}
        {\bfseries Bold extended series}\newline    % the same as \textbf{}
        % The font can be divided into three orthometric dimensions--family, shape and series. 
        % There are two ways to change the font which are font declaration as the examples above. 
            % and box command with parameter as the explaination in annotation. 
        % In latex, available options are given above. 
            % Since some font may not have so all shape or series, 
            % these two command may have different effect from usual font. 
        % Chinese fonts usaully have no other variant, so 
            % different type of fonts are used to replace the variant in a font family. 
            % TODO: See more details in Chinese Documentation. 
        % mixed font
        {\rmfamily\upshape\mdseries Mixed font I}\newline
        \textsf\textit\textbf{Mixed font II}\newline
        % The font's three dimensions can be mixed optionally. 
        % The last slanted letter at the end of a line will cross the right margin. 
            % Use the methods underneath to revise it. 
        {\itshape M}M \quad \textit{M}M \quad {\itshape M\/}M\newline
        \textit{M\nocorr}M\newline         % Forbid the revise. 
        % Also, bold letters may be too close to the puntuation following it. 
            % Underneath are methods to solve it. 
        `{\bfseries leaf}' \quad `{\bfseries leaf\/}' \quad `\textbf{leaf}'\newline
        % EXTENSION: Command \newcommand could be used to reset \nocorrlist{}. 
            % \nocorrlist{} is defualt set with "," and "." in it 
            % which is the same as \newcommand\nocorrlist{,.}
        % If you want to change back to default font in a complex font environment, 
            % do like underneath
        \sffamily\textbf{There is a \textit{complex example, 
        but \textnormal{a normal one} is in} it}\newline
            % {\normalfont ...} is equal to it.  
            % And the font you will get using this command is \rmfamily\mdseries\upshape. 
        % EXTENSION: More details about English font can be found in 
        % Like our dividing font into three dimensions, latex publish its own methods to locate a font, 
            % NFSS(New Font Selection Scheme) which also dividing font into many orthometric properties, 
            % such as encoding, family, series, shape, size and so on. 
        {\fontencoding{OT1}\fontfamily{pzc}\fontseries{m}\fontshape{n}\fontsize{14}{17}
        \selectfont PostScrip New Century Schoolbook}
        

\end{document}