%-*- coding: UTF-8 -*-
%-*- Paragraph.tex -*-
%-*- Paragraph and environment

\documentclass{article}

\usepackage{ragged2e}                   % Hyphenation in other environment. 
\usepackage{lettrine}                   % Drop Capital
\usepackage{shapepar}                   % Paragraph shape
\usepackage{enumitem}                   % Enhance the enumerate
\usepackage{fancyvrb}                   % Use verbatim as a parameter
\usepackage{cprotect}                   % Use verbatim as a parameter
\usepackage{shortvrb}                   % Short \verb
\usepackage{listings}                   % Code
\usepackage{pifont}                     % Circled number
\usepackage{varwidth}                   % varwidth environment
\usepackage{amsfonts}                   % 

\setlength\parskip{0pt}

\newenvironment{myitemize}{%              define list environment myitemize
    \begin{list}{\textbullet}{%
        \setlength\topsep{0pt} \setlength\partopsep{0pt}
        \setlength\parsep{0pt} \setlength\itemsep{0pt}
    }}
    {\end{list}}

\newenvironment{mycenter}{%               define trivlist environment mycenter
    \begin{trivlist}
        \centering\item[]}     
    {\end{trivlist}}

\newtheorem{thm}{Theorem}[section]      % define theorem environmrnt thm

\newsavebox\verbbox                     % define verbbox       
\begin{lrbox}\verbbox
    \verb"#$%^&*()"
\end{lrbox}

\newsavebox\verbatimbox
\begin{lrbox}\verbatimbox
    \begin{minipage}{20em}
        \begin{verbatim}
            #!/bin/sh
            cat ~/${file}$
        \end{verbatim}
    \end{minipage}
\end{lrbox}

\title{Paragraph and Environment}
\author{Wenren Muyan}
\date{\today}

\begin{document}
    \maketitle

    \begin{abstract}
        Paragraph and Environment
    \end{abstract}

    \tableofcontents

    \section{Paragraph}
        \subsection{Indent}
            % The indent at the beginning of a paragrapph
            The first line. \par
            The second line. \par
            \noindent The third line. \par
            % The command \indent is as wide as the \parindent
            \indent The forth line. \par
            \indent\indent The fifth line. \par
        \subsection{Line Spacing}
            % The line spacing is controlled by \parskip which is defaultly set a rubber length, 
            % so that the line spacing can change in a scale in need to make it easy to composing and the article more good-looking. 
            % Use \setlength can change the value of it. 
            % TODO: Put this section in Letter
        \subsection{Alignment}
            % The paragraph defaultly distributes evenly betwwen two margins. 
            % Align on the right.  
            {
                \raggedleft Hello!\par   % The left in the command means the left margin can be not aligned. 
                \raggedleft I am at the right margin. \par
                \raggedleft I am at the right margin. 
            }\par
            % centering
            {
                \centering Hello!\par
                \centering I am in the center!
            }
            % FIXME: Why the second line not align as the command?
            % Alignment environment. 
            \begin{flushright}
                Hello!\par
                I am at the right margin. 
            \end{flushright}
            \begin{center}
                Hello!\par
                I am in the center. 
            \end{center}
        \subsection{Hyphenation}
            % Hyphenation would be used when a long word can not put into the left space of a line. 
            The is a very very very very very very very very 
            long sentence with a longlonglonglong word. \par
            % Use the \- command to tell the compiler where you want to use a hyphenation. 
            The is a very very very very very very very very 
            long sentence with a long\-longlonglong word. 
            % NOTICE: The hyphenation is only used in distributed alignment. 
            % Package Ragged2e can use hypenation in other alignment environment. 
                % The package also provide justify environment and \justifying declaration command to 
                % go back to distributed alignment. 
            % TODO: \sloppy, \fussy
            % Package microtype can adjust the distance between letters in one word to improve the wrapping around in pdfTeX. 
            % TODO: See whether the microtype support XeTeX now in "./rsc/microtype.pdf". 
        \subsection{Paragraph Width}
            % Change the width of a paragraph
            {
                \setlength\leftskip{10cm}
                \setlength\rightskip{2cm}
                The leftskip is the distance to the left margin, 
                and the rightskip is the distance to the right margin. 
            }
            % FIXME: Why no effect?
        \subsection{Paragraph Shape}
            % A special indent is hanging indent, which can be controlled by
                % command \hangindent telling the compiler the indent length,  
                % and \hangafter telling the comiler the amount of the indent lines. 
            {
                \hangindent=5pc \hangafter=-2           % The value of the commands can be positive or negative. 
                Donald Knuth’s TEX, a computerized typesetting system, provides nearly
                everything needed for high-quality typesetting of mathematical notations
                as well as of ordinary text. It is particularly notable for its flexibility, its
                superb hyphenation, and its ability to choose aesthetically satisfying line
                breaks. Because of its extraordinary capabilities, TEX has become the
                leading typesetting system for mathematics, science, and engineering and
                has been adopted as a standard by the American Mathematical Society. 
            }\par
            % Package \lettrine can generate Drop Capital effect. 
            {
                \lettrine{O}ver this period of one hundred years, the CPC has united and 
                led the people in toppling the “three mountains” of imperialism, feudalism 
                and bureaucrat-capitalism, creating the People’s Republic of China (PRC), and 
                completing the New Democratic Revolution and the Socialist Revolution. The 
                political and institutional foundations were thereby laid down to ensure 
                the rights and freedoms of the people. Through successes and setbacks, 
                China has pioneered reform and opening up, set the goal of socialist modernization, 
                and ushered in a new era of building socialism with Chinese characteristics. 
                The Chinese nation has stood up, become better off, and grown in strength. 
                Now, it is embarking on a new journey to build a modern socialist country in all respects.
            }\par
            % TODO: How to let the Drop Cap not in margin?
            % Command \parshape an compose much special shaped paragraph. 
                % See how to use it in 
                % Donald E. Knuth's book "Computers & Typesetting, Volume A: The TeXbook". 
                % or in "./rsc/teximpat.pdf"
            % Package shapepar provide some simple APIs of \parshape, and predefines some shapes that can be directly used. 
            \heartpar{
                Green, green the reed,
                Dew and frost gleam.
                Where’s she I need?
                Beyond the stream.
                Upstream I go,
                The way is long.
                Downstream I go,
                She’s thereamong.
                White, white the reed,
                Dew not yet dried.
                Where’s she I need?
                On the other side.
                Upstream I go,
                Hard is the way.
                Downstream I go,
                She’s far away.
                Bright, bright the reed,
                Dew and frost blend.
                Where’s she I need?
                At river’s end.
                Upstream I go,
                The way does wind.
                Downstream I go,
                She’s far behind.
            }
            % More shape see "./rsc/shapepar.pdf"

    \section{Text Environment}
        % LaTeX has predefined quote, quotation, verse, and abstract text environment. 
        % I have used environment abstract in the beginning of this article. 
            % The abstract environment name is configured by \abstractname. 
            % TODO: How to set your own abstact name?
        In the film \textit{The Shawshank Redemption}, Reid said, 
        \begin{quote}           % no indent
            Some birds are not meant to be caged; their feathers are just too bright. 
        \end{quote}
        There are such paragraphs in \textit{The Communist Party of China and Human Rights Protection -- A 100-Year Quest}
        \begin{quotation}       % indent
            The year 2021 marks the centenary of the Communist Party of China (CPC). 
            Over the past century, the CPC has invested a huge effort in human rights 
            protection, adding significantly to global human rights progress.

            A hundred years ago, the CPC came into being – its mission to salvage the 
            country and save the Chinese people at a perilous time of domestic upheaval 
            and foreign aggression. This was an epoch-changing moment. Under the leadership 
            of the CPC, the Chinese people embarked on a new journey towards prosperity, 
            national rejuvenation, and wellbeing.
        \end{quotation}
        The \textit{Where is She?} is a Chinese classical poemtry in \textit{the Book of Songs}
        \begin{verse}
            Green, green the reed,
            Dew and frost gleam.
            Where’s she I need?
            Beyond the stream.
            Upstream I go,
            The way is long.
            Downstream I go,
            She’s thereamong.
            
            White, white the reed,
            Dew not yet dried.
            Where’s she I need?
            On the other side.
            Upstream I go,
            Hard is the way.
            Downstream I go,
            She’s far away.
            
            Bright, bright the reed,
            Dew and frost blend.
            Where’s she I need?
            At river’s end.
            Upstream I go,
            The way does wind.
            Downstream I go,
            She’s far behind.
        \end{verse}
    
    \section{Enumerate Environment}
        \subsection{Introduction}
            The first situation is: 
            \begin{enumerate}                                           % number
                \item The first
                % The command \item have a parameter to configure personalized number or keyword. 
                \item The second
                \begin{itemize}                                         % no number
                    \item The first
                    \item The second
                    \begin{description}                                 % actually the \item with its parameter
                        \item[1st] One
                        \item[2nd] two
                        \item[3rd] Three
                    \end{description}
                    \item The third
                \end{itemize}
                \item The third
                \item[3a] The A part of the third                       % manual number
                \item[\dag] The most important part of the third        % special symbol
            \end{enumerate}
            % It can be nested in four levels.
        
        \subsection{Enumerate}
            % The number of the enumerate is controlled by a group of counter. 
                % Whenever enter a new enumerate environment, the counter will be reset to 0. 
                % Whenever meet a \item without parameter, the counter plus 1. 
                % There are four default counter in LaTeX, which name are enumi, enumii, enumiii, enumiv
            % Command \the with conuter_name behind will outout the value of the counter. 
                % Actually, the enumerate environment use it to output the number. 
            The first situation is: 
            \begin{enumerate}
                \item Number \theenumi, \labelenumi     % a dot will follow the number with lable command
                \item Number \theenumi, \labelenumi
                \item Number \theenumi, \labelenumi
            \end{enumerate}
            The second situation is: 
            % Other styles of counter number: 
            \begin{enumerate}
                \item Number
                    \arabic{enumi}, \roman{enumi}, \Roman{enumi}, 
                    \alph{enumi}, \Alph{enumi}, \fnsymbol{enumi}
                \item Number
                    \arabic{enumi}, \roman{enumi}, \Roman{enumi}, 
                    \alph{enumi}, \Alph{enumi}, \fnsymbol{enumi}
                \item Number
                    \arabic{enumi}, \roman{enumi}, \Roman{enumi}, 
                    \alph{enumi}, \Alph{enumi}, \fnsymbol{enumi}
            \end{enumerate}
            The third situation is: 
            % Different level of counter will defaultly use different styles of counter number, which are set ahead of time. 
                % We can change the configured style manually
            {
                \renewcommand\theenumi{\roman{enumi}}
                \begin{enumerate}
                    \item The first
                    \item The second
                \end{enumerate}
            }
            % Except the enumerate environment, the pages, sections, and so on also use counter. For example: 
            This sentence is in Page \thepage. 
        \subsection{Custom Counter}
            % It would be needed to define your own counter. 
            % The grammer is like underneath. 
            \newcounter{mycnt}
            \setcounter{mycnt}{0}                                           % set the origin value
            \renewcommand\themycnt{\arabic{mycnt}}                          % set a output command
            \stepcounter{mycnt}The counter's value is \themycnt . \par      % the counter plus 1
            \stepcounter{mycnt}The counter's value is \themycnt . \par
            \addtocounter{mycnt}{2}The counter's value is \themycnt . \par  % the counter plus 2
            \addtocounter{mycnt}{-3}The counter's value is \themycnt . \par   % the counter subtract 1
            % The command \value{counter} will return the value of the certain counter as a parameter. 
            % The \newcounter could be followed by another counter name as its parameter. 
                % That means when the counter used in parameter increases, the newcounter will be reset to zero. 
                % NOTICE: That is how counter subsection, subsubsection... work. 
            % EXTENSION: Package amsmath extends the function of counter, so that it will be more friendly in math equation. 
                % See the usage in "./rsc/amsmath.pdf"
            % Package chngcntr provides function similar to amsmath. 
                % See the usage in "./rsc/chngcntr.pdf"
            % EXTENSION: Further, the counter can be used in complex condition control and loop. 
                % Package ifthen, and calc provide some function of them. 
                % See in "./rsc/ifthen.pdf" and "./rsc/rsc/calc.pdf"
        \subsection{Itemize}
            \begin{itemize}
                \item One \textbullet
                \item Two \textbullet
                \begin{itemize}
                    \item One {\normalfont\bfseries \textendash}
                    \item Two {\normalfont\bfseries \textendash}
                    \begin{itemize}
                        \item One \textasteriskcentered
                        \item Two \textasteriskcentered
                        \begin{itemize}
                            \item One \textperiodcentered
                            \item Two \textperiodcentered
                        \end{itemize}
                    \end{itemize}
                \end{itemize}
            \end{itemize}
            % The \item actually use the \lableitemi, \labelitemii..., 
                % so we can reset the command to another style like enumerate. 
        \subsection{Description}
            % Also, we can modify the style of description environment number label 
                % which defined by the \descriptionlabel
            \renewcommand\descriptionlabel[1]{\normalfont\Large\itshape\textbullet #1}
            \begin{description}
                \item[1st] One
                \item[2nd] Two  
            \end{description}
        \subsection{List}
            % Enumerate, itemize, description are all defined in list environment
                % The list is usually used to define new environment, 
                % of course, it can be used directly with the similar gramer to the definition. 
                % It has many parameters. See the pictures. 
            % TODO: graphic
            % See the difinition in the Introduction
            \begin{myitemize}
                \item One
                \item Two
            \end{myitemize}
            % There is another special environment trivlist can generate a list environment without number label
                % Some environments that seems like irrelevant with enumerate, are usually defined by it. 
                % For example, the center enviroment is defined by it. 
                % The definition is the same as the mycenter in Introduction. 
                % More standard classes see in "./rsc/standardclasses.pdf"
            \begin{mycenter}
                One\par
                Two
            \end{mycenter}
            % The list environment is a little complex to use, 
                % and the enumerate environment configuration is not easy to change, too. 
            % Package enumitem provide methods to use parameters in enumerate. 
            %\begin{enumerate}[%
            %    itemsep=0pt, parsep=0pt, lable=(\arabic*)]
            %    \item One
            %    \item Two
            %\end{enumerate}
            % FIXME: label undifined
                % See the documentation "./rsc/enumitem.pdf"
    \section{Theorem Environment}
        % The theorem environment will generate a title, a number and text with a certain style. 
            % Define the Theorem in Introduction. 
            % The parameter in [] is a counter or other theorem environment which is optional.
        \begin{thm}[the Pythagorean theorem]
            The square of the hypotenuse of a right triangle is equal to 
            the sum of the squares of the two sides. 
        \end{thm}
        \begin{thm}
            The two sides of a triangle add up to more than the third. 
        \end{thm}
        % However, the theorem environment's style is unchangable. 
            % The package theorem and ntheorem provide some comamnds to change the style of theorem environment. 
            % See its usag in "./rsc/theorem.pdf" and "./rsc/ntheorem.pdf". 
            % The package thmtools make it easier to use ntheorem and amsthm
            % See in "./rsc/thmtools.pdf"
    \section{Verbatim}
        % It's difficult to use special symbol or punctuation in LaTeX, especially when you want to use it frequently. 
            % Verbatim can output what you put without regard them as special symbos. 
            % NOTICE: the verbatim will use the typewriter font. 
            % It begin with a symbol and end with a same symbol as the beginner no matter what you use. 
        \verb"\/{}#$%&~!"\par
        \verb!\LaTeX \& \TeX!
        \verb*"1 2  3   4"
        % Verbatim can be also used as an environment. 
        \begin{verbatim}
            #!usr/bin/env perl
            $name = "guy"; 
            print "Hello, $name!\n"; 
        \end{verbatim}
        % The verbatim and other environment can not be used as a parameter in other command. 
            % TODO: Use lrbox environment to save them as a customer box and then use the box as a parameter. 
            % See the verbbox's definition in Introduction. 
        \usebox\verbbox \fbox{\usebox\verbbox}\par
        % Package fancyvrb provides some special commands help us do the above. 
            % Also, fancyvrb extends the verbatim so that we can modify its font, frames, color and so on. 
            % See in "./rsc/fancyvrb.pdf"
        \SaveVerb{myverb}|#$@$%^&|
        \fbox{\UseVerb{myverb}}\par
        % The cprotect do the similar things to the fancyvrb. 
        \cprotect\fbox{\verb|#$@$\%^&|}\par
        % Package verbatim extends the verbatim environment, 
            % which also provide a command \verbatiminput to copy the text in a file you give to. 
            % See in "./rsc/verbatim.pdf"
        % Package can define a short name to \verb. 
        {                               % NOTICE: The symbol you define may conflict with some commands you use behind. 
            \MakeShortVerb"             % use " as the verbatim symbol
            verbatim "\LaTeX"
        }

    \section{Code Environment}
        % Package listings provides various languages environment with code highlight. 
        \lstset{
            numbers=left,
            numberstyle=\footnotesize,
            basicstyle=\sffamily,
            keywordstyle=\bfseries,
            commentstyle=\rmfamily\itshape,
            stringstyle=\ttfamily
        }                                               % The set is optional. 
        \begin{lstlisting}[language=C]
            /* hello.c */
            # include <stdio.h>
            void main(){
                printf("Hello. \n"); 
            }
        \end{lstlisting}
        {
            \lstset{language=C}
            %Use \lstinlilne!typedef char byte!
            % FIXME: How to use inline code?
        }
        % Package listing can also set the color, the background and so on. 
            % See more details in "./rsc/listings.pdf"

    \section{Tabbing Environmrnt}
        % TODO: Page 116
        \begin{tabbing}
            Style\hspace{3em} \= Author \\
            Plain \TeX \> Donald \\
            LaTeX \> Leslie Lamport
        \end{tabbing}
        
        \noindent\hfill$*$\hfill$*$\hfill$*$\hfill

        \begin{tabbing}
            Style\hspace{3em} \= Author \kill
            Plain \TeX \> Donald \\
            LaTeX \> Leslie Lamport
        \end{tabbing}

        % \= set a TAB
        % \' the text front and latter will be centered around the current TAB
        % \` align right with the latter text
        % \< skip to the former TAB
        % \> skip to the latter TAB
        % \+ every line behind this line skips to the right one time
        % \- every line behind this line skips to the left one time
        % \pushtabs save the current TAB
        % \poptabs recover the TAB saved by \pushtabs
        % the common command above in tabbing environment are replaced by \a=, \a', \a` and so on. 
        \newcommand\kw{\textbf}
        \begin{tabbing}
            
            \pushtabs
            Algorithm: Do a binary search for x in list[L, H]. \\
            \qquad\=\+\kw{interger} $L, H, M, J$\\
            \kw{while} \=\+ $L \leq H$ \kw{do} \` $L$ and $H$ are left and right margins. \\
            % FIXME: \lep
                $M \gets \lfloor(L+H)/2\rfloor$ \` $M$ is the center dot. \\
                \kw{case} \=\+\\
                    condition \= foo \+\kill
                    $x > A[M]$: \' $H \gets M-1$ \\
                    $x < A[M]$: \' $H \gets M+1$ \\
                    \kw{else}: \' \= $j \gets M$ \` Find $x$ and return its place. \\
                        \> \kw{return}$(j)$ \\
                \<\< \kw{endcase} \-\-\-\\
            $j \gets 0$\\
            \kw{return}$(j)$ \-\\
            \poptabs
            Example: \\
            $A = \{2, 3, 5, 7, 11\}$, $x = 3$\\
            \qquad\=\+ $M$\qquad \= $L$\qquad \= $H$\qquad \= \\
                        None     \> 1         \> 5          \> initial value, enter the loop \\
                        3        \> 1         \> 2          \> $H$ changes \\
                        2        \> None      \> None       \> find $x$, output place 2. 
        \end{tabbing}

    \section{Box in vertical}
        % The vertical box \vspace{length}, \vfill, \vspace{\fill} is just like the horizontal box command. 
            % See more details in "~/LaTeX/2_Text/2.1_Letter/Letter.tex
        % These command is used in the beginning of the latter line. 
        There is a \parbox{4em}{paragraph box}. \par
        There is another 
        \begin{minipage}{4em}
            paragraph box
        \end{minipage}. \par
        % The paragraph configured like \parindent, \parskip and so on can be used in minipage environment. 
        % Command \parbox and minipage has three optional parameters. 
            % And their total command is \parbox[base_line_position][height][text_position]{width}{text}
            % The base_line_position is the box position and it can use c-center, t-top, b-buttom, which is defaulty c. 
            % The text_position is the text in the box's position, and it can use c, t, b, and s-distributed. 
        \begin{minipage}[c][2.5cm][t]{5em}
            The grass grows  
        \end{minipage}
        \begin{minipage}[c][2.5cm][c]{8em}
            on the edge of a lonely stream,
        \end{minipage}
        \begin{minipage}[c][2.5cm][b]{15em}
            and there is a spring tide with the sound of deep trees. 
        \end{minipage}
        \begin{minipage}[c][2.5cm][s]{14em}
            \setlength\parskip{0pt plus 1pt}
            When showers fall at dusk, the river overflows; \par
            A lonely boat athwart the ferry floats at ease.
        \end{minipage}\par
        % In 5 Verbatim, the \verb can save in the lrbox, while the verbatim environment can not. 
            % We can write the verbatim in minipage, and then save it in lebox. 
            % See the definition in Introduction. 
        \fbox{\usebox\verbatimbox}\quad\fbox{\usebox\verbatimbox}
        % Package fancyvrb also provides \SaveVerbatim and \UseVerbatim to do the above job. 
        % Package varwidth provides varwidth to generate environment with settled max width, 
            % while the width can be changed with your text input. 
        \fbox{
            \begin{varwidth}{10cm}
                Natural \par
                width
            \end{varwidth}
        }
    \section{Page Turning}
        % \raggedbottom means that every page will keep their natural height. 
        % \flushbottom means that all the page will align at the buttom of the paper. 
        % \pagebreak[rank] can skip to the next page with four ranks, 0, 1, 2 and 4, with 4 in force, \nopagebreak is on the contrary. 
        \pagebreak[4]
        % \enlargethispage{length} can increase the height of the page provisionally. 
        \enlargethispage{8em}
        % \newpage can skip to the next page manually. 
            % It is actually equal to the \paragraph and then \vfill the middle text. 
            % It can not work continuously, if it needed, use a \mbox{} in every blank page. 
        % TODO: \clearpage, \cleardoublepage and float. 

    \section{Footnote\protect\footnote{Footnote in the section}}
        This is a footnote\footnote{\textbackslash footnote\{text\}}. \par
        This is another foodnote\footnote[1]{The footnote number is still 1}. \par
        % TODO: use footnote in minipage
        % The footnote uses the counter footnote, and in minipage it uses mpfootnote. 
            % We can change the style of the number of the footnote like the section 3.3. 
        \renewcommand\thefootnote{\fnsymbol{footnote}}
        A symbol number footnote\footnote{}. \par
        \renewcommand\thefootnote{\textcircled{\arabic{footnote}}}
        A circled number footnote\footnote{}. \par
        % The circled number is ugly, use package pifont instead. 
            % \ding outputs the character whose encoding is the number we put.  
            % See more details in "./rsc/psnfss2e.pdf"
        \renewcommand\thefootnote{\ding{\numexpr171+\value{footnote}}}
        A more beautiful circled number foootnote\footnote{}. \par
        % The footnote can not used in box or table. But we could use \footnotemark and \footnotetext instead. 
            % For example
        \begin{tabular}{r|r}
            Independent variable & Dependent varible\footnotemark \\
            \hline
            $x$ & $y$
        \end{tabular}
        \footnotetext{$y=x^2$}
        % Footnote is not allowed to be used in section and annotation directly, 
            % because it is a fragile command. 
            % If you want to use in such environments, use \protect in front like what this section title do. 
        % The line space between footnote is set by \footnotesep. 
        % And the line between text and footnote is set by \footnoterule. 
        % Package footmisc provides more command to configure footnote. 
            % See in "./rsc/footmisc.pdf"
    \section{Margin Paragraph}
        The marginpar will display in the margins\marginpar{Marginpar}. \par
        % The option text will display in the left side margin even number page. 
        Configure the marginpar\marginpar[\hfill Left $\rightarrow$]{$\leftarrow$ Right}
        {
            \reversemarginpar
            % FIXME: Not work?
            The marginpar will display in the left side\marginpar{Marginpar}. 
        }
        % Marginpar is controlled by \marginparwidth, \marginparsep, \marginparpush. 
        % Package geometry, marginnotr, mparhack and endnotes extend the function of marginpar. 


    \section{Rule Box}
        % Rule box is a solid box fill with ink. 
            % \rule[lift_distance]{width}{height}
        \rule{1pt}{1em}Middle\rule{1pt}{1em}\par
        Left\rule[0.5ex]{2cm}{0.6pt}Right\par
        \rule[-0.1em]{1em}{1em}Prove comlete. 

    \section{Strut}
        % Strut is used to be a placeholder, whose width or height is 0. 
            % It is similar to the \phantom
        \fbox{---}\par
        \fbox{\strut---}\par
        \fbox{\rule{0pt}{2em}---}

    \section{Raise Box}
        % Raise box is used to generate a box with lift distance, and we can put our text in it. 
            % That is how the TeX symbol appears. 
            % \raisebox{lift_distance}{height}[depth]{text}
        \mbox{T\hspace{-0.1667em}
            \raisebox{-0.5ex}{E}
            \hspace{-0.125em}X}\quad
            % FIXME: It is not equal to \TeX{}?
        \TeX{}
        
\end{document}