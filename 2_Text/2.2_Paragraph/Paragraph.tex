%-*- coding: UTF-8 -*-
%-*- Paragraph.tex -*-
%-*- Paragraph and environment

\documentclass{article}

\usepackage{ragged2e}                   % Hyphenation in other environment. 
\usepackage{lettrine}                   % Drop Capital
\usepackage{shapepar}                   % Paragraph shape
\usepackage{enumitem}                   % Enhance the enumerate
\usepackage{fancyvrb}                   % Use verbatim as a parameter
\usepackage{cprotect}                   % Use verbatim as a parameter
\usepackage{shortvrb}                   % Short \verb
\usepackage{listings}                   % Code
\usepackage{pifont}                     % Circled number
\usepackage{varwidth}                   % varwidth environment
\usepackage{amsfonts}                   % 

\setlength\parskip{0pt}

\newenvironment{myitemize}{%              define list environment myitemize
    \begin{list}{\textbullet}{%
        \setlength\topsep{0pt} \setlength\partopsep{0pt}
        \setlength\parsep{0pt} \setlength\itemsep{0pt}
    }}
    {\end{list}}

\newenvironment{mycenter}{%               define trivlist environment mycenter
    \begin{trivlist}
        \centering\item[]}     
    {\end{trivlist}}

\newtheorem{thm}{Theorem}[section]      % define theorem environmrnt thm

\newsavebox\verbbox                     % define verbbox       
\begin{lrbox}\verbbox
    \verb"#$%^&*()"
\end{lrbox}

\newsavebox\verbatimbox
\begin{lrbox}\verbatimbox
    \begin{minipage}{20em}
        \begin{verbatim}
            #!/bin/sh
            cat ~/${file}$
        \end{verbatim}
    \end{minipage}
\end{lrbox}

\title{Paragraph and Environment}
\author{Wenren Muyan}
\date{\today}

\begin{document}
    \maketitle

    \begin{abstract}
        Paragraph and Environment
    \end{abstract}

    \tableofcontents

    \section{Paragraph}
        \subsection{Indent}
            The first line. \par
            The second line. \par
            \noindent The third line. \par
            \indent The forth line. \par
            \indent\indent The fifth line. \par
        \subsection{Line Spacing}
            % The line spacing is controlled by \parskip which is defaultly set a rubber length, 
            % so that the line spacing can change in a scale in need to make it easy to composing and the article more good-looking. 
            % Use \setlength can change the value of it. 
            % TODO: Put this section in Letter
        \subsection{Alignment} 
            % Align on the right.  
            {
                \raggedleft Hello!\par   % The left in the command means the left margin can be not aligned. 
                \raggedleft I am at the right margin. \par
                \raggedleft I am at the right margin. 
            }\par
            % centering
            {
                \centering Hello!\par
                \centering I am in the center!
            }
            % FIXME: Why the second line not align as the command?
            % Alignment environment. 
            \begin{flushright}
                Hello!\par
                I am at the right margin. 
            \end{flushright}
            \begin{center}
                Hello!\par
                I am in the center. 
            \end{center}
        \subsection{Hyphenation}
            The is a very very very very very very very very 
            long sentence with a longlonglonglong word. \par
            The is a very very very very very very very very 
            long sentence with a long\-longlonglong word. 
        \subsection{Paragraph Width}
            {
                \setlength\leftskip{10cm}
                \setlength\rightskip{2cm}
                The leftskip is the distance to the left margin, 
                and the rightskip is the distance to the right margin. 
            }
            % FIXME: Why no effect?
        \subsection{Paragraph Shape}
            {
                \hangindent=5pc \hangafter=-2           % The value of the commands can be positive or negative. 
                Donald Knuth’s TEX, a computerized typesetting system, provides nearly
                everything needed for high-quality typesetting of mathematical notations
                as well as of ordinary text. It is particularly notable for its flexibility, its
                superb hyphenation, and its ability to choose aesthetically satisfying line
                breaks. Because of its extraordinary capabilities, TEX has become the
                leading typesetting system for mathematics, science, and engineering and
                has been adopted as a standard by the American Mathematical Society. 
            }\par
            {
                \lettrine{O}ver this period of one hundred years, the CPC has united and 
                led the people in toppling the “three mountains” of imperialism, feudalism 
                and bureaucrat-capitalism, creating the People’s Republic of China (PRC), and 
                completing the New Democratic Revolution and the Socialist Revolution. The 
                political and institutional foundations were thereby laid down to ensure 
                the rights and freedoms of the people. Through successes and setbacks, 
                China has pioneered reform and opening up, set the goal of socialist modernization, 
                and ushered in a new era of building socialism with Chinese characteristics. 
                The Chinese nation has stood up, become better off, and grown in strength. 
                Now, it is embarking on a new journey to build a modern socialist country in all respects.
            }\par 
            \heartpar{
                Green, green the reed,
                Dew and frost gleam.
                Where’s she I need?
                Beyond the stream.
                Upstream I go,
                The way is long.
                Downstream I go,
                She’s thereamong.
                White, white the reed,
                Dew not yet dried.
                Where’s she I need?
                On the other side.
                Upstream I go,
                Hard is the way.
                Downstream I go,
                She’s far away.
                Bright, bright the reed,
                Dew and frost blend.
                Where’s she I need?
                At river’s end.
                Upstream I go,
                The way does wind.
                Downstream I go,
                She’s far behind.
            }

    \section{Text Environment} 
        In the film \textit{The Shawshank Redemption}, Reid said, 
        \begin{quote}           % no indent
            Some birds are not meant to be caged; their feathers are just too bright. 
        \end{quote}
        There are such paragraphs in \textit{The Communist Party of China and Human Rights Protection -- A 100-Year Quest}
        \begin{quotation}       % indent
            The year 2021 marks the centenary of the Communist Party of China (CPC). 
            Over the past century, the CPC has invested a huge effort in human rights 
            protection, adding significantly to global human rights progress.

            A hundred years ago, the CPC came into being – its mission to salvage the 
            country and save the Chinese people at a perilous time of domestic upheaval 
            and foreign aggression. This was an epoch-changing moment. Under the leadership 
            of the CPC, the Chinese people embarked on a new journey towards prosperity, 
            national rejuvenation, and wellbeing.
        \end{quotation}
        The \textit{Where is She?} is a Chinese classical poemtry in \textit{the Book of Songs}
        \begin{verse}
            Green, green the reed,
            Dew and frost gleam.
            Where’s she I need?
            Beyond the stream.
            Upstream I go,
            The way is long.
            Downstream I go,
            She’s thereamong.
            
            White, white the reed,
            Dew not yet dried.
            Where’s she I need?
            On the other side.
            Upstream I go,
            Hard is the way.
            Downstream I go,
            She’s far away.
            
            Bright, bright the reed,
            Dew and frost blend.
            Where’s she I need?
            At river’s end.
            Upstream I go,
            The way does wind.
            Downstream I go,
            She’s far behind.
        \end{verse}
    
    \section{Enumerate Environment}
        \subsection{Introduction}
            The first situation is: 
            \begin{enumerate}                                           % number
                \item The first
                % The command \item have a parameter to configure personalized number or keyword. 
                \item The second
                \begin{itemize}                                         % no number
                    \item The first
                    \item The second
                    \begin{description}                                 % actually the \item with its parameter
                        \item[1st] One
                        \item[2nd] two
                        \item[3rd] Three
                    \end{description}
                    \item The third
                \end{itemize}
                \item The third
                \item[3a] The A part of the third                       % manual number
                \item[\dag] The most important part of the third        % special symbol
            \end{enumerate}
        
        \subsection{Enumerate} 
            The first situation is: 
            \begin{enumerate}
                \item Number \theenumi, \labelenumi
                \item Number \theenumi, \labelenumi
                \item Number \theenumi, \labelenumi
            \end{enumerate}
            The second situation is: 
            % Other styles of counter number: 
            \begin{enumerate}
                \item Number
                    \arabic{enumi}, \roman{enumi}, \Roman{enumi}, 
                    \alph{enumi}, \Alph{enumi}, \fnsymbol{enumi}
                \item Number
                    \arabic{enumi}, \roman{enumi}, \Roman{enumi}, 
                    \alph{enumi}, \Alph{enumi}, \fnsymbol{enumi}
                \item Number
                    \arabic{enumi}, \roman{enumi}, \Roman{enumi}, 
                    \alph{enumi}, \Alph{enumi}, \fnsymbol{enumi}
            \end{enumerate}
            The third situation is: 
            {
                \renewcommand\theenumi{\roman{enumi}}
                \begin{enumerate}
                    \item The first
                    \item The second
                \end{enumerate}
            } 
            % counter page
            This sentence is in Page \thepage. 
        \subsection{Custom Counter}
            \newcounter{mycnt}
            \setcounter{mycnt}{0}                                           % set the origin value
            \renewcommand\themycnt{\arabic{mycnt}}                          % set a output command
            \stepcounter{mycnt}The counter's value is \themycnt . \par      % the counter plus 1
            \stepcounter{mycnt}The counter's value is \themycnt . \par
            \addtocounter{mycnt}{2}The counter's value is \themycnt . \par  % the counter plus 2
            \addtocounter{mycnt}{-3}The counter's value is \themycnt . \par   % the counter subtract 1  
        \subsection{Itemize}
            \begin{itemize}
                \item One \textbullet
                \item Two \textbullet
                \begin{itemize}
                    \item One {\normalfont\bfseries \textendash}
                    \item Two {\normalfont\bfseries \textendash}
                    \begin{itemize}
                        \item One \textasteriskcentered
                        \item Two \textasteriskcentered
                        \begin{itemize}
                            \item One \textperiodcentered
                            \item Two \textperiodcentered
                        \end{itemize}
                    \end{itemize}
                \end{itemize}
            \end{itemize}
             
        \subsection{Description}
            \renewcommand\descriptionlabel[1]{\normalfont\Large\itshape\textbullet #1}
            \begin{description}
                \item[1st] One
                \item[2nd] Two  
            \end{description}
        \subsection{List}
            
            % See the difinition in the Introduction
            \begin{myitemize}
                \item One
                \item Two
            \end{myitemize}

            \begin{mycenter}
                One\par
                Two
            \end{mycenter} 
            %\begin{enumerate}[%
            %    itemsep=0pt, parsep=0pt, lable=(\arabic*)]
            %    \item One
            %    \item Two
            %\end{enumerate}
            % FIXME: label undifined
    \section{Theorem Environment}
        \begin{thm}[the Pythagorean theorem]
            The square of the hypotenuse of a right triangle is equal to 
            the sum of the squares of the two sides. 
        \end{thm}
        \begin{thm}
            The two sides of a triangle add up to more than the third. 
        \end{thm}
    \section{Verbatim}
        \verb"\/{}#$%&~!"\par
        \verb!\LaTeX \& \TeX!\par
        \verb*"1 2  3   4"
        
        \begin{verbatim}
            #!usr/bin/env perl
            $name = "guy"; 
            print "Hello, $name!\n"; 
        \end{verbatim}
            % See the verbbox's definition in Introduction. 
        \usebox\verbbox \fbox{\usebox\verbbox}\par
        \SaveVerb{myverb}|#$@$%^&|
        \fbox{\UseVerb{myverb}}\par
        %  
        \cprotect\fbox{\verb|#$@$\%^&|}\par
        {                               % NOTICE: The symbol you define may conflict with some commands you use behind. 
            \MakeShortVerb"             % use " as the verbatim symbol
            verbatim "\LaTeX"
        }

    \section{Code Environment} 
        \lstset{
            numbers=left,
            numberstyle=\footnotesize,
            basicstyle=\sffamily,
            keywordstyle=\bfseries,
            commentstyle=\rmfamily\itshape,
            stringstyle=\ttfamily
        }                                               % The set is optional. 
        \begin{lstlisting}[language=C]
            /* hello.c */
            # include <stdio.h>
            void main(){
                printf("Hello. \n"); 
            }
        \end{lstlisting}
        {
            \lstset{language=C}
            %Use \lstinlilne!typedef char byte!
            % FIXME: How to use inline code?
        }

    \section{Tabbing Environmrnt}
        \begin{tabbing}
            Style\hspace{3em} \= Author \\
            Plain \TeX \> Donald \\
            LaTeX \> Leslie Lamport
        \end{tabbing}
        
        \noindent\hfill$*$\hfill$*$\hfill$*$\hfill

        \begin{tabbing}
            Style\hspace{3em} \= Author \kill
            Plain \TeX \> Donald \\
            LaTeX \> Leslie Lamport
        \end{tabbing}

        \newcommand\kw{\textbf}
        \begin{tabbing}
            
            \pushtabs
            Algorithm: Do a binary search for x in list[L, H]. \\
            \qquad\=\+\kw{interger} $L, H, M, J$\\
            \kw{while} \=\+ $L \leq H$ \kw{do} \` $L$ and $H$ are left and right margins. \\
                $M \gets \lfloor(L+H)/2\rfloor$ \` $M$ is the center dot. \\
                \kw{case} \=\+\\
                    condition \= foo \+\kill
                    $x > A[M]$: \' $H \gets M-1$ \\
                    $x < A[M]$: \' $H \gets M+1$ \\
                    \kw{else}: \' \= $j \gets M$ \` Find $x$ and return its place. \\
                        \> \kw{return}$(j)$ \\
                \<\< \kw{endcase} \-\-\-\\
            $j \gets 0$\\
            \kw{return}$(j)$ \-\\
            \poptabs
            Example: \\
            $A = \{2, 3, 5, 7, 11\}$, $x = 3$\\
            \qquad\=\+ $M$\qquad \= $L$\qquad \= $H$\qquad \= \\
                        None     \> 1         \> 5          \> initial value, enter the loop \\
                        3        \> 1         \> 2          \> $H$ changes \\
                        2        \> None      \> None       \> find $x$, output place 2. 
        \end{tabbing}

    \section{Box in Vertical} 
        There is a \parbox{4em}{paragraph box}. \par
        There is another 
        \begin{minipage}{4em}
            paragraph box
        \end{minipage}. 

        \begin{minipage}[c][2.5cm][t]{5em}
            The grass grows  
        \end{minipage}
        \begin{minipage}[c][2.5cm][c]{8em}
            on the edge of a lonely stream,
        \end{minipage}
        \begin{minipage}[c][2.5cm][b]{15em}
            and there is a spring tide with the sound of deep trees. 
        \end{minipage}
        \begin{minipage}[c][2.5cm][s]{14em}
            \setlength\parskip{0pt plus 1pt}
            When showers fall at dusk, the river overflows; \par
            A lonely boat athwart the ferry floats at ease.
        \end{minipage}
         
        \fbox{\usebox\verbatimbox}\quad\fbox{\usebox\verbatimbox}

        \fbox{
            \begin{varwidth}{10cm}
                Natural \par
                width
            \end{varwidth}
        }

        \subsection{Rule Box}
        \rule{1pt}{1em}Middle\rule{1pt}{1em}\par
        Left\rule[0.5ex]{2cm}{0.6pt}Right\par
        \rule[-0.1em]{1em}{1em}Prove comlete. 

        \subsection{Strut}
            \fbox{---}\par
            \fbox{\strut---}\par
            \fbox{\rule{0pt}{2em}---}

        \subsection{Raise Box}
            \mbox{T\hspace{-0.49em}
                \raisebox{-0.5ex}{E}
                \hspace{-0.48em}X}\quad
            \TeX{}

        \subsection{Page Turning} 
            \pagebreak[4]
            \enlargethispage{8em}

    \section{Footnote\protect\footnote{Footnote in the section}}
        This is a footnote\footnote{\textbackslash footnote\{text\}}. \par
        This is another foodnote\footnote[1]{The footnote number is still 1}. 
        
        \renewcommand\thefootnote{\fnsymbol{footnote}}
        A symbol number footnote\footnote{}. \par
        \renewcommand\thefootnote{\textcircled{\arabic{footnote}}}
        A circled number footnote\footnote{}. 
       
        \renewcommand\thefootnote{\ding{\numexpr171+\value{footnote}}}
        A more beautiful circled number foootnote\footnote{}. 
    
        \begin{tabular}{r|r}
            Independent variable & Dependent varible\footnotemark \\
            \hline
            $x$ & $y$
        \end{tabular}
        \footnotetext{$y=x^2$}

    \section{Margin Paragraph}
        The marginpar will display in the margins\marginpar{Marginpar}. \par
        % The option text will display in the left side margin even number page. 
        Configure the marginpar\marginpar[\hfill Left $\rightarrow$]{$\leftarrow$ Right}
        {
            \reversemarginpar
            % FIXME: Not work?
            The marginpar will display in the left side\marginpar{Marginpar}. 
        }
        % Marginpar is controlled by \marginparwidth, \marginparsep, \marginparpush. 
        % Package geometry, marginnotr, mparhack and endnotes extend the function of marginpar. 


        
\end{document}