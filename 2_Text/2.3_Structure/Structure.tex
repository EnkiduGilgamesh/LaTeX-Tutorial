%-*- coding: UTF-8 -*-
%-*- Structure.tex -*-
%-*- Article Structure

\documentclass{book}

\usepackage{syntaxonly}             % Check the grammer

\title{Article Structure\thanks{LaTeX}}
\author{Wenren Muyan\thanks{Zhang}\\1st  \and Sam Young\\2nd}
\date{\today}

\includeonly{Chapter.tex}               % Choose files allowed to be compiled. 

%\syntaxonly                % When compiling, only check the grammer, do not ouput the dvi or pdf file

\begin{document}
    % In article the title is not in a single page, while in book or report is in the contrary. 
    % The options titilepage or notitlepage can control it. 

    % LaTeX does not provide methods to change the title page style. 
        % Since the title only uses once, we can change it manually. 
    \frontmatter
    \begin{titlepage}
    \vspace*{\fill}
    \begin{center}
        \normalfont
        {\Huge\bfseries Article Structure}

        \bigskip
        {\Large\itshape Wenren Muyan}

        \medskip
        \today
    \end{center}

    \vspace*{\fill}

    \begin{center}
        {\huge\bfseries Perface}
        
        \vspace{2em}

        This article is about the construct of LaTeX. 
    \end{center}

    \vspace{\stretch{3}}
    
\end{titlepage}

    % The package titling provides methods to change the style through \maketitle. 
        % See in "./rsc/titling.pdf"

    \mainmatter
    \tableofcontents
    
\chapter[Chapter]{7 Different Ranks of Chapter}
    LaTeX has 7 ranks chapters. 
    \begin{tabbing}
        \qquad \=\+ Rank \qquad \= Name \qquad \= Explaination \\
        \vspace{5cm}
                    -1 \>          part \>        the highest rank \\
                    0 \>           chapter \>     report and book's highest rank \\
                    1 \>           section \>     article's highest rank \\
                    2 \>           subsection \\
                    3 \>           subsubsection \\
                    4 \>           paragraoh \>   no number, no display in content \\
                    5 \>           subparaagraph \> no number, no display in content
    \end{tabbing}
% Chapter and section are actually equal to each other. 
% \chapter* means that this chapter no number and no display in content. 
    % The same as the other rank of chapters. 
% \chapter[title_in_content]{title_in_page}

% TODO: counter secnumdepth, tocdepth       % include Chapter.tex
    
\chapter{Nothing}
    There is nothing in this file. 

    It is created to be a test.        % Not in \includeonly file list
                                

    % FIXME: page

    \chapter{Appendix}
        % The chapters in append index will sort by letters and do not show in content. 
            % See in the end of this article. 
            % Package appendix can change the style of the append index, see in "./rsc/appendix.pdf"
    \chapter{Matters}
        % To book, it can be divided into front matter, main matter and back matter. 
            % \frontmatter, \mainmatter, \backmatter
            % The front matter and back matter are no numbers. 

    \chapter{Multiple Files}
        % \include{file_name} will put the code in the selected file to the command position. 
            % Use \includeonly{file_list} to choose file allowed to be compiled. 
        % \input{file_name} copy the code to the command position. Usually be used to input graphics code. 
            % This command does not cause pageskip. 
        The text will be copied into main files. 

\noindent endinpput is an optional command to be the input file's end so that 
the text behind it will not be recognized. 
\endinput

These text will not be copied, they are just like comment. 

    \chapter{Style of Chapter}
        % Use titlesec package to change it. 
            % See in "./rsc/titlesec.pdf"
        % Chinese document see "./zh/Chapter_zh.tex"

\end{document}

